\newpage
\section{Aufgabe 03}
Die dritte Aufgabe besch\"aftigt sich mit dem Odometriefehler.
Wir l\"osen das Problem, indem wir eine Funktion geschrieben haben,
die den Roboter bei dem geradeaus Fahren an den seitlichen W\"anden ausrichtet.
Die Funktion ist abh\"angig von dem Winkel und der Distanz zur Wand.
Sie besteht aus zwei Unterfunktionen, die Funktion, die nur vom Winkel abh\"angt ist eine Tangens-Funktion.
Der Vorteil ist, dass der Roboter, um so schr\"ager zur Wand umso st\"arker parallel zur Wand ausrichtet.
Die Funktion, die abh\"angig von der Distanz ist eine lineare Funktion.
Dadurch richtet sich der Roboter so aus, dass er einen Abstand von 0,4 Meter hat.
Damit ist das Problem der seitlichen Ausrichtung zur Wand gel\"ost.
Das Problem mit dem Ausrichten der vorderen Wand, passiert wenn wir in einer Ecke sind.
