\newpage
\section{Bonusfrage}
"How does your implementation for this assignment (especially the localization part) relate to the content of the lecture (e.g. Bayes filter, particle filter, histogram filter, etc.)?"

Wir haben den aus der Vorlesung bekannten Bayes-Filter zur Lokalisierung des Roboters auf der Karte benutzt. Der Roboter vergleicht seine aktuell wahrnehmbare Umgebung mit den möglichen Positionen und Orientierungen und schließt alle aus, die nicht in Frage kommen. Sollte die Position und Orientierung noch nicht eindeutig sein, bewegt er sich eine Zelle weiter und vergleicht wieder seine Umgebung und schmeißt wieder alle nicht in Frage kommenden Positionen und Orientierungen (der übrig geblibenen) raus. Dies wird wieder so lange wiederholt bis eine eindeutige Position und Orientierung bestimmt wurde. 

Au{\ss}erdem haben wir, f\"ur den Odometriefehler eine Funktion geschrieben, die wir in der Vorlesung besprochen haben.
Daf\"ur war besonders der Teil der Vorlesung n\"utzlich, indem uns erkl\"art wurde, wie man eine solche Funktion erstellt.
Als Ausgangsformel nutzen wir die
$\left(\begin{array}{c} \omega_l \\ \omega_r \end{array}\right) = \frac{1}{r}$
$\left(\begin{array}{rr} 1 & 1 \\ \frac{b}{2} & -\frac{b}{2} \end{array} \right)$
$\left(\begin{array}{c} v \\ \omega \end{array}\right)$
da diese Formel schon von den Einheiten physikalisch korrekt ist.
Es musste nur noch der richtige Winkel zur Wand angegeben werden.
Der erste Schritt bestand darin, Randpunkte zu bestimmen
z.B. welchen Wert der Winkel haben soll, wenn der Roboter ganz nahe an der Wand ist.
Aufgrund der Randpunkte hatten wir uns \"uberlegt, wie die Funktion aussehen soll.
Siehe Aufgabe 3 f\"ur die genaue Funktionsweise.
